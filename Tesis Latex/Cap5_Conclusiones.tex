\chapter{Conclusiones}
%Con este trabajo se pudo observar que el mundo de las redes neuronales y sobre todo las profundas, es un mundo totalmente infinito en el cual existen infinidad de combinaciones entre los hiperpar�metros que estas pose�n, sin embargo y a pesar de esta adversidad, 
Con este trabajo se obtuvo un modelo tanto para la esfericidad como para la redondez el cual es invariante a la escala, rotaci�n, y traslaci�n utilizando la curvatura y el an�lisis de frecuencias en una red neuronal profunda, adem�s de ser mucho m�s r�pido para calcular la redondez.

La red neuronal de la esfericidad por si sola obtuvo un muy buen resultado clasificando de manera muy �ptima el grado de las part�culas, por ser un t�rmino muy simple. Para el caso de la redondez para el caso de 5 clases, se logr� obtener una red la cual arroja que el 70\% de las part�culas de las pruebas estuvo debajo del \(\pm 0.1\), algo que nos interesa porque es la varianza entre las clases y que se puede predecir el 70\% de las veces de manera correcta, adem�s de que est� muy cerca al valor que los geol�gos obtienen al clasificar rocas sedimentarias, que es el 80\%, siendo el nuestro un valor muy aceptable en el campo geol�gico, unos resultados bastante buenos. A medida que se va aumentando la cantidad de clases, el error cada vez va a ser mayor.

Dentro de los trabajos futuros:
\begin{itemize}
	\item Mejorar la arquitectura o las variables de entrada para poder clasificar las 9 clases de redondez.
	\item Buscar un modelo el cual prediga el grado de redondez en n�meros reales.
	\item Seleccionar los coeficientes que m�s aporten a la predicci�n del grado de redondez.
	\item Montar todo el flujo de trabajo dentro de una app o una p�gina web el cual realice el proceso entero desde que se ingresa la imagen de un roca sedimentaria hasta la salida que ser�a su grado de redondez.
	\item Abstraer todo este trabajo para poder predecir el grado de redondez de cualquier objeto, no solamente rocas sedimentarias.
\end{itemize}