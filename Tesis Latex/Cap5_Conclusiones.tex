\chapter{Conclusiones}
%Con este trabajo se pudo observar que el mundo de las redes neuronales y sobre todo las profundas, es un mundo totalmente infinito en el cual existen infinidad de combinaciones entre los hiperpar�metros que estas pose�n, sin embargo y a pesar de esta adversidad, 
En el presente trabajo se obtuvo un modelo para medir la circularidad y la redondez de rocas sedimentaria usando redes neuronales. Este modelo es invariante a la escala, rotaci�n, y traslaci�n. El modelo propuesto utiliza la curvatura y el espectro frecuencial del contorno de la part�cula para determinar el grado de circularidad y redondez.

La red neuronal de la circularidad presento un excelente resultado, con una desviaci�n est�ndar de 0.0176. Para la redondez se logro obtener una red con desviaci�n est�ndar de 0.1014 y una precisi�n de clasificaci�n de 86 \% para un est�ndar geol�gico de 5 clases. Este resultado esta en el orden de clasificaciones realizada por expertos, lo que valida su uso en el campo geol�gico.

Dentro de los trabajos futuros:
\begin{itemize}
	\item Mejorar la arquitectura o las variables de entrada para poder clasificar las 9 clases de redondez.
	\item Buscar un modelo que mejore la estimaci�n del grado de redondez y por lo tanto la clasificaci�n.
	\item Seleccionar los coeficientes que m�s aporten a la predicci�n del grado de redondez.
	\item Montar todo el flujo de trabajo dentro de una app o una p�gina web el cual realice el proceso entero desde que se ingresa la imagen de un roca sedimentaria hasta la salida que ser�a su grado de redondez.
	\item Abstraer todo este trabajo para poder predecir el grado de redondez de cualquier objeto, no solamente rocas sedimentarias.
\end{itemize}