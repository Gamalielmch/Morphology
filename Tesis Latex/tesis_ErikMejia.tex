% ==================================================================
%               FORMATO DE TESIS DE LICENCIATURA UAIE
% ==================================================================
% Este formato de tesis fue diseñado por Manuel Reta Hernández como
% modelo de documento de tesis de licenciatura, propuesto para la
% Unidad Académica de Ingeniería Eléctrica de la Universidad Autónoma
% de Zacatecas.
% El archivo utiliza el paquete uaztesism.cls y fue adaptado de los
% paquetes de tesis de "the Purdue University" y de "the University of
% Wisconsin-Madison".

% ==================================================================
% INICIA PREÁMBULO DEL DOCUMENTO
% ==================================================================

% -----------DECLARACIÓN DE TIPO DE DOCUMENTO A DISEÑAR-------------

\documentclass[12pt,tesis]{uaztesis}

% --------------DECLARACIÓN DE PAQUETES A UTILIZAR------------------
%(se pueden agregar a la lista cuantos paquetes sean necesarios)
\usepackage[latin1]{inputenc}
\usepackage[nottoc]{tocbibind}
\usepackage{amssymb,amsmath,amsfonts}
\usepackage{color,graphicx,curves}
\usepackage{subfigure}
\usepackage{multirow,rotating}
\usepackage[T1]{fontenc}
\usepackage{hyperref}
\usepackage{float}


% --DECLARACIÓN DE RUTA DE CARPETA DONDE SE ENCUENTRAN LAS FIGURAS--
\graphicspath{{Figuras/}}

% --------------OPCIONES DE IMPRESIÓN DE FIGURAS--------------------
% Con la instrucción \psdraft se crea un espacio en blanco en donde van
% las figuras. Esto ahorra toner en la impresión del borrador de tesis.
%
% Con la instrucción \psfull se incluyen a todas las figuras.
\psfull

% --------------DECLARACIÓN DEL ESTILO DE PÁGINA--------------------
% El estilo de página \pagestyle{thesis} coloca los encabezados
% correctos de la versión final del documento.

\pagestyle{thesis}
%\noappendixfigures
\noappendixtables


% ===================================================================
% INICIA EL CONTENIDO DEL DOCUMENTO
% ===================================================================

\begin{document}

% Para opción de tesis de licenciatura
\thesis

% Introducción en sílabas de palabras desconocidas por LaTeX
% Declaración de número de páginas en números Romanos
\clearpage\pagenumbering{roman}

% ------------------INTRODUCCIÓN DE DATOS----------------------------

% Título de la tesis, autor, y grado a recibir:
\title{Estimaci\'on de par\'ametros morfol\'ogicos en rocas sedimentarias usando Fourier el�ptico y redes neuronales}
\author{Erik Mej\'ia Hern\'andez}
\degree{Maestr\'ia en Ciencias del Procesamiento de la Informaci\'on}
\carrera{Ingenier\'ia El\'ectrica}

% Grado y nombre de asesores de tesis:
\advisortitle{Dr.} \advisorname{Jos\'e de Jes\'us Villa Hern\'andez}
\gradoasesor2{Dr.} \nombreasesor2{Gamaliel Ch\'avez Moreno}

% Fecha de solicitud de tema de tesis:
\fechasol{12 de septiembre de 2008}

% Fecha de aprobación de tema de tesis:
\fecharev{18 de octubre de 2008}

%Fecha de autorización de impresión de tesis:
\fechaaut{19 de enero de 2009}

% Fecha de Examen profesional:
\date{Algun dia de 2020}

% Grado y nombre de director de la Unidad Académica:
\dirtitle{M. en C.} \dirname{José Manuel Cervantes Viramontes}

% Grado y nombre de los tres vocales restantes, miembros del jurado:
\vocalbtitle{Dr.} \vocalbname{Jorge de la Torre y Ramos}
\vocalctitle{Ing.} \vocalcname{Amando Castañeda Carrillo}
\vocaldtitle{M. en A.} \vocaldname{Manuel Haro Macías}

% ---------GENERACIÓN DE PÁGINAS PRELIMINARES DEL DOCUMENTO----------

% Genera página de presentación
\maketitle

% Genera oficio de autorización de tema de tesis
%\begin{revision}
%\end{revision}

% Genera oficio de autorización de impresión de tesis
%\begin{autorizacion}
%\end{autorizacion}

% Genera oficio de aprobación del Examen
%\begin{aprobacion}
%\end{aprobacion}

% Genera páginas de resumen
%\begin{resumenespa}
%  \input{resumen}              % incluye al archivo resumen.tex
%\end{resumenespa}

% Genera página de dedicatoria (opcional)
%\begin{dedicatoria}
%  \input{dedicatoria}          % incluye al archivo dedicatoria.tex
%\end{dedicatoria}

% Genera página de agradecimientos (opcional)
%\include{agradecimientos}      % incluye al archivo agradecimientos.tex

% Genera páginas del contenido general y lista de figuras y tablas
\tableofcontents
\renewcommand{\listfigurename}{Lista de Figuras}
\listoffigures                 % se indica sólo si hay figuras

%\listoftables                  % se indica sólo si hay tablas

% Genera páginas del nomenclatura
%\include{nomenclatura}    % incluye al archivo nomenclatura.tex

%--------INCLUSIÓN DE LOS ARCHIVOS (CAPÍTULOS) DEL DOCUMENTO--------

% Declaración de número de páginas en números Arábigos
\clearpage\pagenumbering{arabic}

\chapter{Introducci�n}

\section{Antecedentes}

Nuestro planeta est� conformado por la hidrosfera, atmosfera, biosfera y tierra s�lida. El componente principal de la tierra s�lida son las rocas (Tarbuck, 2005).  Las rocas son agregados naturales de uno o m�s minerales. Estas pueden clasificarse por su origen y proceso en tres clases: �gneas, metam�rficas y sedimentarias (J�rgen, 2015). Las rocas �gneas son las que se forman a partir del enfriamiento de minerales fundidos (magma) entre la corteza terrestre y el manto superior. Las rocas �gneas algunas veces pueden alcanzar la parte superior de la  corteza terrestre por medio de volcanes o por el ascenso de capas de la corteza. En la corteza existe un proceso llamado meteorizaci�n que consiste en la fragmentaci�n de rocas por alteraciones f�sicas y qu�micas (como la gravedad, erosi�n, materia org�nica). Estas rocas se transportan generalmente por gravedad y se depositan en las zonas m�s bajas de la corteza terrestres (la mayor�a en los oc�anos). Estos sedimentos son nuevas rocas y se les conocen como rocas sedimentarias. Las rocas metam�rficas se generan a partir de rocas �gneas, sedimentarias o mismas rocas metam�rficas. Como su nombre lo indica estas rocas se generan por el cambio (metamorfosis) de una roca madre, este cambio es generado por altas presiones y temperaturas, pero sin lleguen a fundirse (Tarbuck, 2005). 

De estos tres tipos de rocas, las m�s importantes son las rocas sedimentarias por las siguientes razones: (1) representan el 80\% de la corteza terrestre, (2) permiten conocer los procesos e historia de la tierra, (3) son de gran importancia en el sector econ�mico porque de ellas derivan el petr�leo, gas natural, carb�n, sal, azufre, potasio, yeso, caliza, fosfato, uranio y m�s minerales (Folk, 1980), (4) en algunos casos representan un riesgo para poblaciones como la asentadas en las cercan�as de volcanes o grandes sedimentos, (5) en el estudio de suelo para la construcci�n (Rodriguez, et. al., 2014).  

Las rocas sedimentarias se estudian por su composici�n f�sica, qu�mica  y mineral�gica. El estudio f�sico se conforma por tres par�metros; tama�o, morfolog�a y orientaci�n. El conocer estos par�metros nos permite deducir el origen, los diversos procesos transporte, el entorno reol�gico y clim�tico y su deposici�n. Para medici�n de tama�o y la orientaci�n existen diversas t�cnicas muy bien establecidas y muy precisas (Tucker, 2009). Por otro lado la morfolog�a es un concepto reciente, en comparaci�n a los otros y a�n se encuentra en desarrollo y b�squeda de conceptos universales (Diepenbroek, 1992).  

La morfolog�a describe la forma (shape) de objetos o part�culas mediante mediciones de su contorno. La morfolog�a no s�lo es importante en el estudio de rocas sedimentarios sino que se extienda a otros campos cient�ficos y productivos como la nanomedicina, agricultura, biolog�a, neurociencias, arte visual, entre otros (Fontoura and Marcondes, 2009, Samar, et. al., 2011, Randall, et. al., 2014).  ha sido y es una rama muy importante de nuestra vida, ya que ella se ha encargado de entender y estudiar la raz�n del porque tienen cierto aspecto externo todos y cada uno de los objetos o seres vivos. La morfolog�a de rocas sedimentarias se describe por tres par�metros: forma general (form), redondez (roundness) y textura superficial (roughness), los cuales se relacionan a procesos geol�gicos. Estos tres par�metros son jer�rquicos y de escalas diferentes, por lo que uno no afecta al otro. La forma es la caracter�stica de mayor jerarqu�a que est� relacionada con los aspectos m�s generales. La forma se calcula mediante relaciones axiales adimensionales o relaciones de circularidad. La redondez es una caracter�stica intermedia superpuesta a la forma. El grado de redondez o angularidad est� relacionado con las curvas y las esquinas principales del contorno. La rugosidad o textura se refiere a irregularidades m�s finas superpuestas en la redondez y la forma (Barrett, 1980; Blott y Pye, 2008; Powers, 1953). Estas propiedades se muestran en la Figura 1.1.

\begin{figure}[H]
	\centering 
	\includegraphics[scale=.2]{figuras/fig1.png}
	\caption{Forma, redondez y  textura superficial  propuestas por Barrett (1980).}
	\label{fig1}
\end{figure}

Existen diversas expresiones para medir forma, una de las m�s usadas en el campo geol�gico es la propuesta por Wadell (1935), la cual se obtiene de la relaci�n entre el radio del c�rculo cuya �rea es igual a la de la part�cula y el radio del c�rculo m�s peque�o que inscribe a la part�cula (Wadell, 1935). Existen tres enfoques para medir la redondez; los basados en curvatura, los que emplean Fourier y los relacionados con Fractales.  El m�todo basado en curvatura es simple y preciso, sin embargo es un m�todo que depende de la escala. Los m�todos basados en Fourier son muy populares sin embargo analizar el espectro es complicado y de un alto costo computacional. El uso de fractales para describir la forma se ha vuelto popular sin embargo tiene problemas para identificar algunos tipos de redondez y son muy sensibles al suavizado de contornos.

En la presente tesis nos planteamos usar redes neuronales para estimar la forma y redondez de rocas sedimentarias. La variable de entrada a la red neuronal es el PCA del espectro de Fourier el�ptico. Se eligi� esta variable por ser invariante a la escala, la rotaci�n y traslaci�n.  Como objetivo para la forma se emple� la circularidad propuesta por Wadell (1935) descrita anteriormente. Para la redondez, se eligi� como objetivo el grado de angulosidad calculado con el m�todo propuesto por Wadell (1933), el cual define el grado de redondez como la relaci�n entre el radio de curvatura promedio de las esquinas de una part�cula y el radio del c�rculo circunscrito m�s grande posible. La red neuronal utilizada tiene la siguiente arquitectura: red neuronal de 6 capas, la capa de entrada con 50 neuronas y funci�n de activaci�n ReLU, 4 capas ocultas con 50 neuronas cada una, con funci�n de activaci�n ReLU. La capa de salida con una sola neurona con funci�n de activaci�n lineal. La base de datos para entrenar la red neuronal se compone de 1000 im�genes de rocas reales de diversos fen�menos geol�gicos. La red neuronal tiene un error de m�nimos cuadrados de 7.1708e-04 con los datos de entrenamiento. El resultado fue comparado con clasificaciones visual realizadas por Pettijohn y Krumbein. La red neuronal nos permite tener la redondez y la circularidad en tiempo 2800 veces m�s r�pido que el m�todo de Wadell (1935), adem�s de ser invariante a la escala, rotaci�n y traslaci�n. 


\section{Planteamiento del problema de investigaci�n}
\label{Sec12}

Los m�todos para medir la forma general y redondez no son invariantes a la escala, rotaci�n y traslaci�n. Los m�todos basados en Fourier y fractales son invariantes a estas 3 transformaciones, pero el tratamiento de su resultado es complejo. Por lo que no existe un m�todo invariante y f�cil de ajustar.


\section{Justificaci�n del problema de investigaci�n}

El an�lisis morfol�gico de las rocas sedimentarias es importante en geolog�a para la reconstrucci�n hist�rica de nuestro planeta. Tambi�n es importante en sectores econ�micos y de prevenci�n de riesgo. Por lo que es necesario tener un m�todo el cual sea invariante a la escala, rotaci�n y traslaci�n, as� como preciso, f�cil y r�pido de usar. 


\section{Preguntas de Investigaci�n}
\begin{itemize}
\item �Cu�l ser� el desempe�o del modelo basado en red neuronales en comparaci�n con el algoritmo de (Hryciw, 2016) que usa el m�todo de (Wadell,1933)?
\item �Cu�les son los mejores rangos de arm�nicos para predecir la redondez y la circularidad utilizando un modelo basado en redes neuronales?
\end{itemize}

\section{Objetivo General}
Obtener un modelo basado en redes neuronales para clasificar la Forma general (Wadell, 1935) y redondez (Wadell, 1933) de las rocas sedimentarias, utilizando el espectro de Fourier el�ptico como entrada. Con el fin de tener una herramienta precisa, f�cil y r�pida que pueda ser usada para fines geol�gicos.

\section{Objetivos Espec�ficos}

\begin{enumerate}
\item Estudiar y aplicar la circularidad propuesto por Wadell (1935). 
\item Estudiar y aplicar la redondez propuesto por Wadell (1933) utilizando el algoritmo de c�rculos circunscritos de Hryciw (2016).
\item Estudiar y aplicar el m�todo de Fourier El�ptico propuesto por Giardina (1981).
\item Entrenar la red neuronal con el PCA de los arm�nicos de Fourier El�ptico (Giardina, 1981), usando la redondez y circularidad a la que pertenecen como salida.
\item Contrastar los resultados de la red neuronal con los que se obtienen utilizando los objetivos espec�ficos 1 y 2.

\end{enumerate}

\section{Hip�tesis}
La red neuronal mide la redondez con mayor velocidad y precisi�n que el algoritmo de Hryciw (2016).

\section{Estructura de la tesis}
La estructura de esta tesis va a estar distribuida en 5 cap�tulos, los cuales son:
\begin{itemize}

\item Introducci�n: Se explica el contenido a grandes rasgos de este trabajo.
\item Marco te�rico: Se exponen las teor�as base del trabajo, como a su vez la comparaci�n de los trabajos relacionados contra el planteamiento en este.
\item M�todo y propuesta de investigaci�n: Se explican a fondo los m�todos que se usaron durante el trabajo.
\item Resultados y limitaciones: Se desarrollan los casos de prueba que se usaron para la experimentaci�n y se exponen las limitaciones que se tienen.
\item Conclusiones: Se habla si la hip�tesis fue cumplida, y si los objetivos propuestos para este trabajo fueron alcanzados o no.
	
\end{itemize}


       % incluye al archivo introduccion.tex
\chapter{Marco Te�rico}
\section{La tierra como sistema}
La geolog�a, ciencia de la tierra, estudia nuestro planeta como un sistema que engloba cuatro esferas; hidrosfera, atmosfera, biosfera y tierra s�lida. En la figura 2.1 se muestran ejemplos ilustrativos de cada esfera. La hidrosfera es una masa de agua din�mica que est� en movimiento continuo, evapor�ndose de los oc�anos a la atm�sfera, precipit�ndose sobre la Tierra y volviendo de nuevo al oc�ano por medio de los r�os. La atm�sfera  es una capa gaseosa que rodea a la Tierra. A pesar de sus modestas dimensiones, este delgado manto de aire es una parte integral del planeta. No s�lo proporciona el aire que respiramos, sino que tambi�n nos protege del intenso calor solar y de las peligrosas radiaciones ultravioletas. La biosfera incluye toda la vida en la Tierra. Est� concentrada cerca de la superficie en una zona que se extiende desde el suelo oce�nico hasta varios kil�metros de la atm�sfera. Debajo de la atm�sfera y los oc�anos se encuentra la Tierra s�lida. Gran parte del estudio de la Tierra s�lida se concentra en los eventos geogr�ficos superficiales (m�s accesibles). Por fortuna, estos eventos externos se relacionan directamente con lo que ocurre debajo de la superficie. Examinando los rasgos superficiales m�s destacados y su extensi�n global, podemos obtener pistas para explicar los procesos din�micos que han conformado nuestro planeta Se dice que es un sistema debido a que estas cuatro esferas interact�an constantemente (\cite{Lutgens2005};\cite{rafferty2011geological}).

\begin{figure}[H]
	\centering 
	\includegraphics[scale=1]{figuras/geologia.png}
	\caption{Ejemplo de las cuatro esferas de la tierra: hidrosfera, atm�sfera, biosfera y tierra s�lida.}
\end{figure}

La tierra solida se divide en tres capas la corteza, el manto y el n�cleo.  La corteza, capa rocosa externa, comparativamente fina de la Tierra, se divide generalmente en corteza oce�nica y corteza continental. El Manto representa m�s del 82 por ciento del volumen de la Tierra est�, una envoltura rocosa s�lida que se extiende hasta una profundidad de 2.900 kil�metros. El l�mite entre la corteza y el manto representa un cambio de composici�n qu�mica. N�cleo. Se cree que la composici�n del n�cleo es una aleaci�n de hierro y n�quel con cantidades menores de ox�geno, silicio y azufre, elementos que forman f�cilmente compuestos con el hierro (\cite{Lutgens2005}; Raffereti, 2012). En la Figura 2.2 se detalla las subdivisiones de las  tres capas.

\begin{figure}[H]
	\centering 
	\includegraphics[scale=1]{composicion_tierra.png}
	\caption{Estructura en capas de la Tierra.}
\end{figure}

De todas las capas existe un alto inter�s por la corteza ya que en ella se desvuelve la vida humana y la biosfera en general. Esta superficie terrestre es divida en continentes y las cuencas oce�nicas. Estas dos superficies tienen diferentes caracter�sticas f�sicas y qu�micas. Para la superficie terrestre continental el elemento fundamental de mayor abundancia son las rocas. Al examinar una roca con atenci�n, encontramos que consta de cristales o granos m�s peque�os denominados minerales. Los minerales son compuestos qu�micos (o en algunas ocasiones elementos �nicos), cada uno de ellos con su propia composici�n y sus propiedades f�sicas. La naturaleza de la roca est� definida por su composici�n qu�mica y por sus propiedades texturales (tama�o, forma y orientaci�n). Estos dos aspectos son reflejo de los procesos geol�gicos que la crearon. Esta comprensi�n tiene muchas aplicaciones pr�cticas, como en la b�squeda de recursos minerales y energ�ticos b�sicos y la soluci�n de problemas ambientales. 

\section{Las rocas y su ciclo}

Los ge�logos dividen las rocas en tres grandes grupos: �gneas, sedimentarias y metam�rfica, algunos ejemplos de cada tipo de roca se muestran en la Figura 2.3. Rocas �gneas. Las rocas �gneas (ignis = fuego) se forman cuando la roca fundida, denominada magma, se enfr�a y se solidifica. El magma es roca fundida que se puede formar a varios niveles de profundidad en el interior de la corteza de la Tierra y el manto superior. A medida que se enfr�a el magma, se van formando y creciendo los cristales de varios minerales. Cuando el magma permanece en el interior profundo de la corteza, se enfr�a lentamente durante miles de a�os. Las rocas �gneas de grano grueso que se forman muy por debajo de la superficie se denominan plut�nicas. Las rocas �gneas que se forman en la superficie terrestre se denominan volc�nicas y suelen ser de grano fino. 

\begin{figure}[H]
	\centering 
	\includegraphics[scale=1.1]{tiposderocas.png}
	\caption{Ejemplos de los tres tipos de rocas.}
\end{figure}

Las rocas sedimentarias se forman por acumulaci�n de sedimentos, en la figura 2.4 se muestra un tipo de sedimento. Los sedimentos est�n compuestos de part�culas de diversos tama�os transportadas y son sometidos a procesos f�sicos y qu�micos (diag�nesis), que dan lugar a materiales consolidados. El agua, el viento o el hielo glacial suelen transportar los productos de la meteorizaci�n (fragmentaci�n de rocas) a lugares de sedimentaci�n donde �stos forman capas relativamente planas. Normalmente los sedimentos se convierten en roca o se litifican por la compactaci�n y cementaci�n. La compactaci�n tiene lugar a medida que el peso de los materiales suprayacentes comprime los sedimentos en masas m�s densas. La cementaci�n se produce conforme el agua que contiene sustancias disueltas se filtra a trav�s de los espacios intergranulares del sedimento. Con el tiempo, el material disuelto en agua precipita entre los granos y los cementa en una masa s�lida.

El tercer tipo son las rocas metam�rficas. Estas se producen a partir de rocas �gneas, sedimentarias o incluso otras rocas metam�rficas. As�, cada roca metam�rfica tiene una roca madre, la roca a partir de la que se ha formado. Metam�rfico es un adjetivo adecuado porque su significado literal es �cambiar la forma�. La mayor�a de cambios tienen lugar a temperaturas y presiones elevadas que se dan en la profundidad de la corteza terrestre y el manto superior. Los procesos que crean las rocas metam�rficas a menudo progresan de una manera incremental, desde cambios ligeros (metamorfismo de grado bajo) hasta cambios sustanciales (metamorfismo de grado alto).


\begin{figure}[H]
	\centering 
	\includegraphics[scale=1.1]{rocasedimentaria.png}
	\caption{Sedimento ubicado en M�tlili, en la provincia de Gharda�a, Argelia.}
\end{figure}

Estos tres tipos de rocas interact�an, de hecho forman un ciclo, v�ase la 2.5. Las rocas pueden pasar por cualquiera de los tres estados cuando son forzadas a romper el equilibrio. Una roca �gnea como el basalto puede disgregarse y alterarse cuando se expone a la atm�sfera, o volver a fundirse al subducir por debajo de un continente. Debido a las fuerzas generadoras del ciclo de las rocas, las placas tect�nicas y el ciclo del agua, las rocas no pueden mantenerse en equilibrio y son forzadas a cambiar ante los nuevos ambientes. 

Podemos iniciar explicando el ciclo con el magma. El magma que es la roca fundida que se forma a una gran profundidad por debajo de la superficie de la Tierra. Con el tiempo, el magma se enfr�a y se solidifica. Este proceso, denominado cristalizaci�n, puede ocurrir debajo de la superficie terrestre o, despu�s de una erupci�n volc�nica, en la superficie. En cualquiera de las dos situaciones, las rocas resultantes se denominan rocas �gneas.

\begin{figure}[H]
	\centering 
	\includegraphics[scale=1.1]{ciclo_rocas.png}
	\caption{Ciclo de las rocas.}
\end{figure}

Si las rocas �gneas afloran en la superficie experimentar�n meteorizaci�n, en la cual la acci�n de la atm�sfera desintegra y descompone lentamente las rocas. Los materiales resultantes pueden ser desplazados pendiente abajo por la gravedad antes de ser captados y transportados por alg�n agente erosivo como las aguas superficiales, los glaciares, el viento o las olas. Por fin, estas part�culas y sustancias disueltas, denominadas sedimentos, son depositadas. Aunque la mayor�a de los sedimentos acaba llegando al oc�ano, otras zonas de acumulaci�n son las llanuras de inundaci�n de los r�os, los desiertos, los pantanos y las dunas. A continuaci�n, los sedimentos experimentan litificaci�n, un t�rmino que significa �conversi�n en roca�. El sedimento suele litificarse dando lugar a una roca sedimentaria cuando es compactado por el peso de las capas suprayacentes o cuando es cementado conforme el agua subterr�nea de infiltraci�n llena los poros con materia mineral. Si la roca sedimentaria resultante se entierra profundamente dentro de la tierra e interviene en la din�mica de formaci�n de monta�as, o si es intruida por una masa de magma, estar� sometida a grandes presiones o a un calor intenso, o a ambas cosas. La roca sedimentaria reaccionar� ante el ambiente cambiante y se convertir� en un tercer tipo de roca, una roca metam�rfica. Cuando la roca metam�rfica es sometida a cambios de presi�n adicionales o a temperaturas a�n mayores, se fundir�, creando un magma, que acabar� cristalizando en rocas �gneas. Los procesos impulsados por el calor desde el interior de la Tierra son responsables de la creaci�n de las rocas �gneas y metam�rficas. La meteorizaci�n y la erosi�n, procesos externos alimentados por una combinaci�n de energ�a procedente del Sol y la gravedad, producen el sedimento a partir del cual se forman las rocas sedimentarias. Caminos alternativos. Las v�as mostradas en el ciclo b�sico no son las �nicas posibles. Al contrario, es exactamente igual de probable que puedan seguirse otras v�as distintas de las descritas en la secci�n precedente. Esas alternativas se indican mediante las l�neas azules en la Figura 2.5. Las rocas �gneas, en vez de ser expuestas a la meteorizaci�n y a la erosi�n en la superficie terrestre, pueden permanecer enterradas profundamente. Esas masas pueden acabar siendo sometidas a fuertes fuerzas de compresi�n y a temperaturas elevadas asociadas con la formaci�n de monta�as. Cuando esto ocurre, se transforman directamente en rocas metam�rficas. Las rocas metam�rficas y sedimentarias, as� como los sedimentos, no siempre permanecen enterrados. Antes bien, las capas superiores pueden ser eliminadas, dejando expuestas las rocas que antes estaban enterradas. Cuando esto ocurre, los materiales son meteorizados y convertidos en nueva materia prima para las rocas sedimentarias. Las rocas pueden parecer masas invariables, pero el ciclo de las rocas demuestra que no es as�. Los cambios, sin embargo, requieren tiempo; grandes cantidades de tiempo.   

\section{Rocas sedimentarias}

Las rocas sedimentarias son de gran inter�s por las siguientes razones: (1) Cubren alrededor del 80\% de la corteza terrestre, que es la parte de la tierra s�lida con la que m�s interactuamos. (2) Representan la base del conocimiento de otras �reas geol�gicas como la estratigraf�a y la geolog�a estructural. (3) Un alto porcentaje de la actividad econ�mica est� relacionada con dep�sitos de rocas sedimentarias, algunos ejemplos son: el petr�leo, el gas natural, carb�n, sal, sulfuro, potasio, yeso, caliza, fosfato, uranio, hierro, magnesio y una numerosa lista de elementos en la construcci�n. (4) En la geotecnia son una parte importante para caracterizar el tipo de suelo. (5) En la petrolog�a sedimentaria es clave para determinar la litolog�a, relieve, clima y actividad tect�nica.     

Como se describi� anteriormente las rocas sedimentarias son resultante del dep�sito de material solido producto de la meteorizaci�n mec�nica y qu�mica. La transformaci�n del sedimento en roca sedimentaria se conoce como  litificaci�n. El sedimento puede experimentar grandes cambios desde el momento en que fue depositado hasta que se convierte en una roca sedimentaria y posteriormente es sometido a las temperaturas y las presiones que lo transforman en una roca metam�rfica. El t�rmino diag�nesis (dia =cambio; genesis=origen) es un t�rmino general para todos los cambios qu�micos, f�sicos y biol�gicos que tienen lugar despu�s de la deposici�n de los sedimentos, as� como durante y despu�s de la litificaci�n. La litificaci�n se refiere a los procesos mediante los cuales los sedimentos no consolidados se transforman en rocas sedimentarias s�lidas (lithos =piedra; fic =hacer). Los procesos b�sicos de litificaci�n son la compactaci�n y la cementaci�n, v�ase la figura 2.6.

\begin{figure}[H]
	\centering 
	\includegraphics[scale=1.1]{procesosRocas.png}
	\caption{Formaci�n de rocas sedimentarias. Litificaci�n de sedimentos. Compactaci�n y cementaci�n.}
\end{figure}

El cambio diagen�tico f�sico m�s habitual es la compactaci�n. Conforme el sedimento se acumula a trav�s del tiempo, el peso del material suprayacente comprime los sedimentos m�s profundos. Cuanto mayor es la profundidad a la que est� enterrado el sedimento, m�s se compacta y m�s firme se vuelve. Al inducirse cada vez m�s la aproximaci�n de los granos, hay una reducci�n considerable del espacio poroso (el espacio abierto entre las part�culas). Conforme se reduce el espacio del poro, se expulsa gran parte del agua que estaba atrapada en los sedimentos. Dado que las arenas y otros sedimentos gruesos son s�lo ligeramente compresibles, la compactaci�n, como proceso de litificaci�n, es m�s significativa en las rocas sedimentarias de grano fino.  La cementaci�n es el proceso m�s importante mediante el cual los sedimentos se convierten en rocas sedimentarias. Es un cambio diagen�tico qu�mico que implica la precipitaci�n de los minerales entre los granos sedimentarios individuales. Los materiales cementantes son transportados en soluci�n por el agua que percola a trav�s de los espacios abiertos entre las part�culas. A lo largo del tiempo, el cemento precipita sobre los granos de sedimento, llena los espacios vac�os y une los clastos. De la misma manera que el espacio del poro se reduce durante la compactaci�n, la adici�n de cemento al dep�sito sedimentario reduce tambi�n su porosidad.

El sedimento tiene dos or�genes principales. En primer lugar, el sedimento puede ser una acumulaci�n de material que se origina y es transportado en forma de clastos s�lidos derivados de la meteorizaci�n mec�nica y qu�mica. Los dep�sitos de este tipo se denominan detr�ticos y las rocas sedimentarias que forman, rocas sedimentarias detr�ticas. En la Figura 2.7 se muestran algunos tipos 

\begin{figure}[H]
	\centering 
	\includegraphics[scale=1.1]{rocasSedimentariasDetriticas.png}
	\caption{Rocas sedimentarias detr�ticas.}
\end{figure}

La segunda fuente principal de sedimento es el material soluble producido en gran medida mediante meteorizaci�n qu�mica. Cuando estas sustancias disueltas son precipitadas mediante procesos org�nicos o inorg�nicos, el material se conoce como sedimento qu�mico y las rocas formadas a partir de �l se denominan rocas sedimentarias qu�micas.

\begin{figure}[H]
	\centering 
	\includegraphics[scale=1.1]{rocasSedimentariasQuimicas.png}
	\caption{Rocas sedimentarias qu�micas.}
\end{figure}

\section{Morfolog�a de rocas sedimentarias}

El an�lisis de las rocas sedimentarias contempla aspectos f�sicos, qu�micos y mineral�gicos. Dentro de los f�sicos existen tres descriptores; el tama�o, la morfolog�a y la orientaci�n. El tama�o y la orientaci�n de la roca es un tema bien establecido. Sin embargo la morfolog�a es un tema relativamente reciente y se encuentra en desarrollo. 

En la morfolog�a de la roca se graban la configuraci�n de la g�nesis, el ambiente, transporte y deposici�n. Por ejemplo, cuando las corrientes de agua, el viento o las olas mueven la arena y otros clastos sedimentarios, los granos pierden sus bordes y esquinas angulosos y se van redondeando m�s a medida que colisionan con otras part�culas durante el transporte. Por tanto, es probable que los granos redondeados hayan sido transportados por el aire o por el agua. Adem�s, el grado de redondez indica la distancia o el tiempo transcurrido en el transporte del sedimento por corrientes de aire o agua. Granos muy redondeados indican que se ha producido una gran abrasi�n y, por consiguiente, un prolongado transporte. Los granos muy angulosos, por otro lado, significan dos cosas: que los materiales sufrieron transporte durante una distancia corta antes de su dep�sito, y que quiz� los haya transportado alg�n otro medio. Por ejemplo, cuando los glaciares mueven los sedimentos, los clastos suelen volverse m�s irregulares por la acci�n de trituraci�n y molienda del hielo. Adem�s de afectar al grado de redondez y al grado de selecci�n que los clastos experimentan, la duraci�n del transporte a trav�s de corrientes de agua y aire turbulentas influye tambi�n en la composici�n mineral de un dep�sito sedimentario. Una meteorizaci�n sustancial y un transporte prolongado llevan a la destrucci�n gradual de los minerales m�s d�biles y menos estable. As� en la morfolog�a esta parte de la clave para reconstruir la historia y caracter�sticas de procesos sedimentarios.  


\section{Forma, redondez y rugosidad}

La morfolog�a de una roca sedimentaria puede ser descrita por tres componenetes: la forma, la redondez y la rugosidad. La Figura 2.9 ilustra las tres componentes. Estas propiedades son independientes entre s�, esto es que una puede variar sin afectar las otras dos. Est�s tres propiedades se distinguidas, al menos, por sus diferentes escalas. Esta caracter�stica permite ordenarlas de  manera jerarquica, como se muestra en la Figura 2.10. La forma, propiedad de primer orden, refleja los grandes rasgos que tiene la part�cula; la redondez, propiedad de segundo orden, refleja los cambios en las esquinas. Estas variaciones se encuentran superpuestas en la forma. La rugosidad, propiedad de tercer orden, son las variaciones superpuestas en la superficie y en las esquinas [\cite{BARRETT1980}].

\begin{figure}[H]
	\centering 
	\includegraphics[scale=1]{Forma,redondez,rugosidad.png}
	\caption{Una representaci�n simplificada de la forma, redondez y rugosidad en 3 dimensiones para ilustrar su independencia [\cite{BARRETT1980}].}
\end{figure}

Este modo jer�rquico de la forma, redondez y rugosidad esta basado en los fenomenos geol�gico a los que se exponen las rocas. Cambios en la rugosidad no necesariamente afectan a la redondez. La meteorizaci�n puede aumentar la rugosidad de una roca, pero las esquinas muy redondeas se mantendr�n igual. Estr�as, quebraduras y otras caracter�sticas se pueden obtener sin cambiar la redondez. Esto no imposibilita que este proceso haga que la rugosidad cambie la redondez despu�s de un largo per�odo de tiempo. La redondez de una roca, puede incrementar a trav�s de la abrasi�n, sin afectar mucho a la forma. En contraste, un cambio en la forma inevitablemente afectar� a la redondez y rugosidad, porque las superficies nuevas son expuestas, y aparecer�an nuevas esquinas, y un cambio en la redondez deber�a afectar a la rugosidad, as� que por cada cambio resulta en una nueva morfolog�a [\cite{BARRETT1980}].

\begin{figure}[H]
	\centering 
	\includegraphics[scale=.2]{figuras/fig1.png}
	\caption{Forma, redondez y  textura superficial  propuestas por \cite{BARRETT1980}.}
\end{figure}


Existen definiciones y clasificaciones bien definidas para medir la forma y redondez. Por otro lado, en lo que respecta a la rugosidad no existe un acuerdo entre los cientificos, sin embargo hay algunos intentos de caracterizarla y medir [\cite{Rodriguez2013}].

El presente trabajo se enfoca a la medici�n de la forma y la redondez, dejando de lado la rugosidad por no existir a�n un consenso en su descripci�n. Los diferentes m�todos para medir la forma y redondez se discutir�n en los siguientes dos subsecciones.


\section{M�todos para obtener la forma}

El parametro m�s util para medir la forma es el grado de esfericidad. La esfericidad cuantifica el grado de similitud entre la part�cula y una esfera. Una de las m�tricas m�s usadas para medir la esfericidad es la propuesta por \cite{Wadell1932}, que define la esfericidad como la relaci�n entre el �rea de superficie de la esfera del mismo volumen y el �rea de la superficie real de la part�cula. Esta m�trica necesita las tres dimensiones de la part�culas, sin embargo existe su equivalente bidimensional [\cite{Altuhafi2013}].

Las siguientes m�tricas representan las formas m�s comunes para medir la esfericidad en 2 dimensiones son:

\begin{equation}
\text{Esfericidad por el �rea} \colon S_A = \frac{A_s}{A_{cir}}
\end{equation}
 donde \(A_s\) es el �rea proyectada de la part�cula y \(A_{cir}\) es el �rea del c�rculo m�nimo que inscribe a la part�cula, f�rmula propuesta por \cite{tickell1939examination}.

\begin{equation}
\text{Esfericidad por el di�metro} \colon S_D = \frac{d_c}{D_{cir}}
\end{equation}

 donde \(d_c\) es el di�metro del c�rculo igual en �rea a la de la part�cula en el tama�o original cuando descansa en una de sus caras m�s largas, m�s o menos paralela a el plano con los di�metros m�s largos e intermedios, y \(D_{cir}\) es el di�metro del c�rculo m�s peque�o que inscribe a la part�cula en su tama�o original. Los rangos de valor de est� f�rmula est�n distribuidos de 0.54 a 1, f�rmula propuesta por \cite{Wadell1935}.

\begin{equation}
\text{Esfericidad por la relaci�n de c�rculos} \colon S_C = \frac{D_{ins}}{D_{cir}}
\end{equation}

 donde \(D_{ins}\) es el di�metro del m�nimo c�rculo circunscrito en la part�cula y \(D_{cir}\) es el di�metro del c�rculo m�s peque�o que inscribe a la part�cula, f�rmula propuesta por \cite{Santamarina2004}.

\begin{equation}
\text{Esfericidad por el Per�metro} \colon S_P = \frac{P_c}{P_s}
\end{equation}

donde \(P_c\) es el per�metro del c�rculo que tiene la misma �rea que la part�cula y \(P_s\) es el per�metro de la part�cula, f�rmula propuesta por \cite{Altuhafi2013}.

\begin{equation}
\text{Esfericidad por relaci�n entre anchura y altura} \colon S_{WL} = \frac{d_1}{d_2}
\end{equation}

donde \(d_1\) y \(d_2\) son el ancho y el largo de la part�cula respectivamente, f�rmula propuesta por \cite{krumbein1951stratigraphy}.

Existen m�s definiciones para la esfericidad que se basan en el volumen, como la propuesta por \cite{wadell1933sphericity}, pero no se toman en cuenta porque en este trabajo las part�culas que ser�n caracterizadas son en dos dimensiones.

Para fines de este trabajo, la esfericidad que se usa es la propuesta por \cite{Wadell1935} para 2 dimensiones, debido a que es una de las m�s conocidas y usadas para describir la forma de una part�cula.

\section{M�todos para obtener la redondez}
La redondez es un concepto que suele confundirse con propiedades de la forma, es por eso que su concepto debe ser aclarado (\cite{krynine19561}; \cite{sneed1958pebbles}). Como se mencion� anteriormente, la redondez es una propiedad superpuesta a la forma que estima  la suavidad (o angulosidad) superficial de la part�cula. [\cite{Resentini2018}]

Las tres formas m�s comunes para medir la redondez se basan en c�rculos circunscritos, espectro de Fourier, y fractales.

El m�todo basado en c�rculos circunscritos parte de que el radio de curvatura de una esquina no puede ser mayor al radio del m�ximo c�rculo circunscrito en la part�cula, por lo que la redondez de una esquina puede ser expresada como \(\frac{r}{R}\), donde \(r\) es el radio de curvatura de la esquina y \(R\) es el radio del m�ximo c�rculo circunscrito. [\cite{Wadell1932}]

La redondez total de un s�lido en un solo plano (2 dimensiones) se obtiene usando la media de la redondez de cada una de las esquinas en ese plano. Por lo que la f�rmula queda as�:

\begin{equation}
\frac{\sum{\frac{r}{R}}}{N} = \text{Grado de redondez}
\end{equation}

donde \(\sum{\frac{r}{R}}\) es la sumatoria de los valores de redondez de cada esquina, y \(N\) es el n�mero de esquinas de la part�cula en el plano dado. El m�ximo valor de redondez que puede ser obtenido es de \(1\), f�rmula propuesta por \cite{Wadell1932}. 

El m�todo que se basa en el espectro de Fourier parte de que se puede descomponer todas las frecuencias que posee una imagen de una part�cula, obteniendo claramente cuales pertenecen a la redondez. 

\cite{ehrlich1970exact} propusieron el uso de la transformada de Fourier para medir la rugosidad,

\begin{equation}
P = [\frac{1}{2}\sum_{u=1}^{N-1}{[R^2(u)+I^2(u)]}]^{\frac{1}{2}}
\end{equation}

El m�todo de \cite{DIEPENBROEK1992} esta basado en el coeficiente de rugosidad (Eq. (2.7)). Este m�todo fue probado usando los  \(N=5,6,...,60\) arm�nicos de la transformada de Fourier. Obteniendo que los primeros 23 arm�nicos fueron los que dieron un mejor resultado para caracterizar la redondez de una part�cula. [\cite{Drevin2006}]

\begin{figure}[H]
	\centering 
	\includegraphics[scale=.6]{figuras/wadellRoundness.png}
	\caption{Clasificaciones para la redondez de una part�cula propuestas por \cite{Wadell1932}.}
\end{figure}

Desde un punto de vista pr�ctico, no es suficiente solo medir el grado de redondez sino tambi�n clasificarla. Una herramienta muy usada por los sedimentologos para clasificar la morfolog�a de clastos es un gr�fico visual comparativo. Russell, Taylor y Pettijohn (M�ller,1967) desarroll� un gr�fico visual comparativo de un conjunto de referencias de part�culas de la redondez conocidas. Este gr�fico ofrec�a un manera f�cil y r�pida para estimar la redondez de part�culas en dos dimensiones. The Russell, Taylor y Pettijohn (RTP) referenciaban la figura con 25 part�culas organizadas en 5 diferentes categor�as de redondez; angular, sub-angular, sub-redondeada, redondeada y bien redondeada (Boggs,2012).


\begin{figure}[H]
	\centering 
	\includegraphics[scale=.5]{figuras/krumbeinRoundness.png}
	\caption{Clasificaciones para la redondez de una part�cula propuestas por \cite{Krumbein1941}.}
\end{figure}

En la figura 2.11 se observan las clasificaciones propuestas , proponiendo 6 clases, mientras que la clasificaci�n de figura 2.12 propone 9 clases.

El m�todo a utilizar para la redondez va a ser el propuesto por \cite{Wadell1932}, debido a su simplicidad con respecto a los m�todos basados en Fourier y Fractales, a su vez, la escala que se utilizar� ser� la de \cite{Krumbein1941} por el hecho de ser m�s interesante de trabajar con m�s clases.

\section{Modelo o esquema general de investigaci�n}
Este trabajo tiene un enfoque de investigaci�n de tipo Aplicada, ya que busca la manera de crear un modelo de redes neuronales junto con el espectro de Fourier el�ptico de las rocas sedimentarias. El modelo que se busca debe de ser preciso, r�pido y f�cil de usar, adem�s, que sea invariante a la escala, rotaci�n y traslaci�n.
 













































%\section{Modelos Ocultos de Markov}
%A continuacion abordaremos la teor�a de los Modelos Ocultos de Markov, que ha demostrado
%ser la que mejores resultados produce a la hora de implementar reconocedores de voz. Los introducimos
%a partir de los procesos m�s sencillos, a los que iremos a�adiendo elementos para obtener los
%modelos finales. Gran parte de este tema se dedica a resolver anal�ticamente los problemas
%fundamentales que se presentar�n al intentar aplicar estas ideas al reconocimiento de voz.
%
%\subsection{Procesos de Markov}
%\subsubsection{Definici�n}
%Un proceso de Markov es un proceso aleatorio {q (n)} discreto en el tiempo, con la particularidad
%de que la probabilidad del valor de cada muestra s�lo depende del valor de la anterior:
%
%\begin{figure}[H]
%\centering 
%\includegraphics[scale=.7]{mark.PNG}
%%\caption{Diagrama preprocesado de la se�al}
%\label{fig3}
%\end{figure}
%
%Se usar� la notaci�n [q\_{n}] = q (n) para denotar al estado del proceso en el instante n o muestra
%n -�sima , pues resume la informaci�n que toda la historia del proceso aporta al futuro del
%mismo.
%
%\textbf{Ejemplo}
%Un modelo AR(1) (Autorregresivo de orden 1) es un proceso de Markov:
%
%\begin{figure}[H]
%\centering 
%\includegraphics[scale=.7]{mark2.PNG}
%%\caption{Diagrama preprocesado de la se�al}
%\label{fig3}
%\end{figure}
%
%\subsection{Probabilidad de una secuencia concreta}
%Para calcular ahora la probabilidad de una secuencia concreta de muestras, debemos primero obtener el siguiente resultado v�lido para cualquier proceso aleatorio discreto.
%
%Sea {q=o... hasta la e-nesima} una secuencia concreta, obtenida al observar el proceso aleatorio
%{ } n q durante T muestras. Utilizando el Teorema de Bayes, podemos descomponer la probabilidad
%de obtener esa secuencia en t�rminos de un producto de las probabilidades de cada muestra,
%condicionadas por las muestras pasadas:
%
%\begin{figure}[H]
%\centering 
%\includegraphics[scale=.7]{mark3.PNG}
%%\caption{Diagrama preprocesado de la se�al}
%\label{fig3}
%\end{figure}
%
%Siempre se puede descomponer de la siguiente forma:
%\begin{figure}[H]
%\centering 
%\includegraphics[scale=.7]{mark4.PNG}
%%\caption{Diagrama preprocesado de la se�al}
%\label{fig3}
%\end{figure}
%
%Si seguimos descomponiendo, se puede inducir: 
%\begin{figure}[H]
%\centering 
%\includegraphics[scale=.7]{mark5.PNG}
%%\caption{Diagrama preprocesado de la se�al}
%\label{fig3}
%\end{figure}
%
%Los modelos anteriormente vistos son demasiado restrictivos para aplicarlos a una gran variedad
%de problemas de inter�s, puesto que cada estado se corresponde con un evento f�sico observable.
%Podemos extender el concepto de modelo de Markov para incluir el caso en el que la observaci�n
%es aleatoria, dependiente del estado en el que se encuentra el sistema.
%El modelo resultante se conoce como Modelo Oculto de Markov (HMM), y es un proceso
%doblemente estoc�stico con:
%
%\begin{itemize}
%\item Un proceso estoc�stico subyacente que no es observable (est� oculto), sino que s�lo puede
%ser observado a trav�s de otro proceso (que s� es observable). Conforma la secuencia de
%estados por la que pasa el sistema.
%\item Un conjunto de procesos estoc�sticos que producen la secuencia de observaci�n.
%\end{itemize}
%
%Por lo tanto, un Modelo Oculto de Markov es una cadena de Markov en la que la observaci�n no
%es la propia secuencia de estados (que permanece oculta), sino que es el resultado de ciertos
%procesos estoc�sticos que se producen en cada estado.
%
%\subsection{Elementos de un Modelo Oculto de Markov}
%
%Un modelo oculto de Markov esta caracterizado por: 
%
%\begin{itemize}
%\item El n�mero de estados en el modelo (N ). Aunque los estados est�n ocultos, para muchas
%aplicaciones pr�cticas hay un significado f�sico asociado a los estados del modelo.
%Normalmente los estados est�n interconectados tal que cada uno puede ser alcanzado
%desde cualquier otro estado (modelo erg�dico). Sin embargo, como veremos m�s tarde,
%existen otros tipos de conexi�n entre estados que resultar�n de inter�s en el caso
%concreto del reconocimiento de voz.
%
%\item El n�mero de s�mbolos distintos observables en cada estado (M), es decir, el tama�o
%del alfabeto. Aqu�, los s�mbolos observados corresponden con la salida f�sica del sistema.
%
%\item La distribuci�n de probabilidades de transici�n entre estados ( A ).
%
%\item La distribuci�n de probabilidades de los s�mbolos observados en el estado.
%
%\item La distribuci�n inicial de estados.
%\end{itemize}
%
%\subsubsection{Representacion mediante rejilla}
%
%Podemos a�adir elementos a la representaci�n del trellis para modelar HMMs:
%
%\begin{itemize}
%\item Probabilidades iniciales de cada estado.
%\item Probabilidades de observaci�n de s�mbolos en cada estado.
%\end{itemize}
%
%\begin{figure}[H]
%\centering 
%\includegraphics[scale=.8]{trellis.PNG}
%%\caption{Diagrama preprocesado de la se�al}
%\label{fig3}
%\end{figure}
%
%\section{Modelado del Habla para el reconocimiento de voz}
%
%\subsection{Fundamentos}
%Para el reconocimiento de palabras aisladas, los modelos ocultos de Markov han demostrado en
%aplicaciones pr�cticas ser aquellos que mejores resultados producen. La idea es asignar a cada
%posible palabra del vocabulario (de tama�o W ) un modelo de estructura similar pero de
%diferentes par�metros:
%
%\begin{figure}[H]
%\centering 
%\includegraphics[scale=.7]{mark6.PNG}
%%\caption{Diagrama preprocesado de la se�al}
%\label{fig3}
%\end{figure}
%
%Cada modelo est� formado por distintos estados, en los cuales se generan los s�mbolos que
%constituyen la secuencia de observaci�n. Por otro lado, cada palabra est� formada por distintos
%fonemas. La relaci�n entre estados y fonemas no es directa y ser� abordada m�s adelante. La
%secuencia de estados es desconocida a priori:
%
%\begin{figure}[H]
%\centering 
%\includegraphics[scale=.7]{mark7.PNG}
%%\caption{Diagrama preprocesado de la se�al}
%\label{fig3}
%\end{figure}
%
%Por �ltimo la palabra recibida (y que queremos reconocer) se corresponder�, despu�s de un
%an�lisis que extrae sus caracter�sticas m�s importantes (MFCC), con la secuencia observada. �sta
%es conocida en las etapas de entrenamiento, validaci�n y reconocimiento.
%
%\subsection{Procedimiento para el modelado}
%
%Consideremos el siguiente reconocedor de palabras aisladas. Para cada palabra de un vocabulario
%(de W palabras) queremos dise�ar un HMM diferente, de N estados. Representamos la se�al de
%voz de una palabra dada como una secuencia discreta de vectores que codifican las
%caracter�sticas esenciales de la palabra: los coeficientes MFCC.
%
%La primera tarea es construir modelos individuales para cada palabra, usando la soluci�n al
%problema 3 para estimar los par�metros �ptimos.
%
%Para la comprensi�n del significado f�sico de los estados del modelo, usamos la soluci�n al
%problema 2. Con esto conseguimos segmentar cada modelo en estados, y entonces estudiar las
%propiedades de los vectores de observaci�n. El objetivo es refinar el modelo para mejorar su
%capacidad de modelar la secuencia recibida.
%
%Finalmente, una vez el conjunto de W modelos han sido dise�ados y optimizados, el
%reconocimiento de una palabra se realiza usando la soluci�n al problema 1. Se asigna una
%puntuaci�n a cada modelo basada en la secuencia observada de entrada, y se elige como palabra
%reconocida aquella cuya puntuaci�n sea m�xima.
%
%En la pr�ctica, la verosimilitud puede computarse a trav�s de la secuencia de estados m�s
%probable (problema 2), sin recurrir al c�lculo minucioso de todas las combinaciones posibles
%(problema 1).
%
%
%
%\section{Trabajos Relacionados}
%
%Dentro de la literatura que se ha estado recolectando para la escritura de esta tesis se ha encontrado un gran avance en cuanto a los Sistemas de Reconocimiento Autom�tico del Habla, los cuales se listan a continuaci�n y mas adelante ser�n abordados cada uno de ellos mas profundamente y obtener la informaci�n mas relevante que ayude al desarrollo de este trabajo de tesis. 
%
%\begin{itemize}
%\item Algoritmos y M�todos para el Reconocimiento de Voz en Espa�ol Mediante S�labas (2006)
%\item Reconocimiento de Voz Usando HTK (Universidad de Sevilla)
%\item Sistema de control activado por voz para uso en domotica (Enero 2016)
%\item Aplicaciones en reconocimento de voz utilizando HTK (Bogota, 2005)
%\item Modelo Ac�stico y de Lenguaje del Idioma Espa?nol para el dialecto Cucute?no, Orientado al Reconocimiento Automatico del Habla (2017)
%\item Reconomiento de Voz de ni�os para el espa�ol hablado en Mexico usando SONIC
%\item Un programa de reconocimiento de voz ayuda a los ni�os a aprender a leer (2004)
%\item El reconocimiento de voz mejora el rendimiento escolar de los ni�os con dislexia (2017)
%\item Implementaci�n de un m�dulo de reconocimiento de voz para ni�os mediante el procesamiento de se�ales aplicado en un caso pr�ctico (2017)\
%\item Efectos de un sistema de reconocimiento de voz en la conciencia fonol�gica y las
%habilidades de lectura de ni�os preescolares espa�oles (2016)
%\item Usabilidad en sistemas l�dicos infantiles con reconocimiento de voz como apoyo en la terapia de
%rehabilitaci�n de ni�os con problemas de lenguaje (2008)
%\item Reconocimiento de Voz para Ni�os con Discapacidad en el Habla (2004)
%
%\end{itemize}





       % incluye al archivo cuerpo_tesis.tex
\chapter{Modelo y propuesta de Investigaci�n}

\section{Modelo de Investigaci�n}
El trabajo de investigaci�n est� dividido en 6 etapas de las cuales se van a describir a continuaci�n:
\begin{itemize}
\item La primera etapa consiste en conseguir 1500 im�genes de todas las clases de esfericidad y redondez para poder entrenar de manera equitativa la red, y despu�s probar con las im�genes de (Krumbein, 1941) y verificar los resultados.
\item La segunda etapa se obtiene el valor de redondez y esfericidad de cada una de las im�genes de entrenamiento con los m�todos propuestos.
\item La tercera etapa se saca el valor de las constantes de los primeros 40 arm�nicos de la serie de Fourier el�ptico de cada una de las im�genes de entrenamiento, aparte, relacionar estos arm�nicos con su valor de esfericidad y redondez.
\item La cuarta etapa consiste entrenar las 2 redes neuronales, la que va a clasificar la esfericidad, y la que va a clasificar la redondez, con los valores de entrada que ser� los arm�nicos de Fourier el�ptico y su respectiva salida.
\item Una vez entrenada cada red, obtener los arm�nicos de Fourier el�ptico de las im�genes de prueba.
\item La �ltima etapa es clasificar las im�genes de prueba y observar los resultados.
\end{itemize}

\begin{figure}[H]
	\centering 
	\includegraphics[scale=1]{figuras/modeloInvestigacion.png}
	\caption{Modelo de Investigaci�n}
	\label{fig1}
\end{figure}

\section{Fourier el�ptico}
El m�todo de los c�rculos circunscritos para medir la redondez de una imagen (Hryciw, 2016) es para encontrar el c�rculo que mejor se ajusta en cada parte del contorno, con ciertas restricciones para ser tomado en cuenta o no, y al final poder relacionar todos esos c�rculos con el c�rculo circunscrito mayor de la figura, y poder obtener el valor de redondez que posee (Figura 1). (Se seleccion� a partir del grado de curvatura).

\begin{figure}[H]
	\centering 
	\includegraphics[scale=.7]{figuras/metodoCirculos.png}
	\caption{Exposici�n del resultado del m�todo de c�rculos circunscritos.}
	\label{fig1}
\end{figure}

Se decidi� usar por encima de otros porque es el uso del m�todo propuesto por (Krumbein, 1941) para medir la redondez pero llevado de forma digital, adem�s de ser un m�todo muy bien aceptado en el ambiente de la geolog�a.

Fourier el�ptico (GIARDINA, 1981) es un m�todo en el cual se descompone la se�al de una imagen en una sumatoria infinita de t�rminos, donde cada t�rmino es una elipse con sus caracter�sticas espec�ficas, independientes de las dem�s, haciendo m�s f�cil la tarea de analizar dicha informaci�n por separado, y presuntamente encontrar patrones que dif�cilmente o con m�s trabajo se podr�an encontrar analizando la imagen completa.

La decisi�n de usar Fourier el�ptico por encima de Fourier es el hecho de que no puede describir la esfericidad de una figura elongada (Figura 1) con tan pocos arm�nicos, por el hecho de usar una sumatoria infinita de c�rculos (Por estar trabajando en 2 dimensiones), en cambio Fourier el�ptico solo le bastar�an unos cuantos arm�nicos para hacerlo, como su nombre lo dice, usa elipses para aproximar la figura, por lo que resulta en una reducci�n de informaci�n para describir absolutamente lo mismo, adem�s de ser invariante a la escala y rotaci�n, este hecho nos permite no preocuparnos que los valores resultantes cambien debido a que las im�genes est�n escaladas, o, giradas.

\begin{figure}[H]
	\centering 
	\includegraphics[scale=1]{figuras/particulaElongada.png}
	\caption{Imagen blanco y negro de una roca sedimentaria elongada.}
	\label{fig1}
\end{figure}

La decisi�n de usar las redes neuronales (NN) como la forma para poder clasificar nuestros datos fue por el hecho de la capacidad que tienen para aprender solas con los ejemplos que se le dan, pueden usarse para clasificar m�ltiples clases, y la versatilidad que tienen para configurarse de manera interna, est�s fueron las razones por las que se usaron NN, pero al final se podr�a cuestionar el hecho de usar NN en vez de las redes neuronales convolucionales (CNN), ya que estamos trabajando en el dominio de las im�genes y las CNN son excelente para ello, bueno, el hecho de usarlas requiere que el par�metro de entrada sea totalmente toda la imagen resultando en que absolutamente cada una tenga que tener el mismo tama�o, por lo que se opt� en buscar una forma en la cual se pueda sustituir esa entrada y ah� es cuando entra Fourier el�ptico, como se mencion� en el p�rrafo anterior, b�sicamente se va a entregar la misma informaci�n de la imagen pero de una manera distinta y con menos par�metros de entrada.

\section{Algoritmo para estimar la redondez}
\section{Redes neuronales}

         % incluye al archivo Ecuaciones.tex
%
\chapter{Casos de estudió}

\section{Resultados Preliminares}

Resultados de la clasificación de la esfericidad de 1125 partículas.
\begin{figure}[H]
	\centering 
	\includegraphics[scale=1]{figuras/resultadosEsfericidad.png}
	\caption{Contraste de la esfericidad obtenidas por la red neuronal contra el valor real.}
\end{figure}

Resultados de la clasificación de la esfericidad de 1125 partículas.
\begin{figure}[H]
	\centering 
	\includegraphics[scale=1]{figuras/resultadosRedondez.jpg}
	\caption{Contraste de la redondez obtenidos por la red neuronal contra el valor real.}
\end{figure} % incluye al archivo Ecuaciones.tex
%\include{Conclusiones}            % incluye al archivo Conclusiones.tex

% ---------------------INTRODUCCIÓN DE APÉNDICES--------------------
% usar \begin{appendix}..\end{appendix} para sólo un apéndice
% usar \begin{appendices}..\end{appendices} para varios apéndices

%\begin{appendices}
%%\include{apenA}                   % incluye al archivo apendA.tex
%%\include{apenB}                   % incluye al archivo apendB.tex
%\end{appendices}

% ------------------INTRODUCCIÓN DE REFERENCIAS---------------------
%\begin{thebibliography}[99]
%\medskip

\printbibliography[title={Referencias}]
%\bibliography{bibliografia}               % incluye al archivo 
%\end{thebibliography}

% ===================================================================
% FINALIZA EL CONTENIDO DEL DOCUMENTO
% ===================================================================

\end{document}